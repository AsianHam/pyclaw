%
%  PyClaw 5.0 Design Document
%
%  Created by Kyle Mandli on 2011-03-29.
%  Copyright (c) 2011 University of Washington. All rights reserved.
%
\documentclass[]{article}

% Use utf-8 encoding for foreign characters
\usepackage[utf8]{inputenc}

% Setup for fullpage use
\usepackage{fullpage}

% Uncomment some of the following if you use the features
%
% Running Headers and footers
%\usepackage{fancyhdr}

% Multipart figures
%\usepackage{subfigure}

% More symbols
\usepackage{amsmath}
\usepackage{amssymb}
\usepackage{latexsym}

% Surround parts of graphics with box
\usepackage{boxedminipage}

% Package for including code in the document
\usepackage{listings}

% If you want to generate a toc for each chapter (use with book)
\usepackage{minitoc}

% This is now the recommended way for checking for PDFLaTeX:
\usepackage{ifpdf}

\ifpdf
\usepackage[pdftex]{graphicx}
\else
\usepackage{graphicx}
\fi

% Commands 
\include{commands}

% Graphics
\DeclareGraphicsRule{.tif}{png}{.png}{`convert #1 `dirname #1`/`basename #1 .tif`.png}
\graphicspath{{/Users/mandli/Documents/Research/figures/},{./figures/}}
\usepackage{epstopdf}

\title{PyClaw 5.0 Design Document}
\author{Kyle T. Mandli et al}
\date{}

\begin{document}

\ifpdf
\DeclareGraphicsExtensions{.pdf, .jpg, .tif}
\else
\DeclareGraphicsExtensions{.eps, .jpg}
\fi

\makesmalltitle{PyClaw 5.0 Design Document}{Kyle T. Mandli et al.}

\section{Goals of PyClaw}
\begin{enumerate}
    \item Provide a interface to multiple packages with a unified front end
    \item Provide an easy to use teaching tool for students wishing to learn these types of methods
    \item Provide a platform for the exploration of methods
\end{enumerate}
\section{Structures}

\subsection{Solution Structures}

\subsection{Solver Structures}

\section{Proposed Interfaces}

\subsection{Solution}


\end{document}
