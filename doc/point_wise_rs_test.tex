%
%  Riemann Solver Comparison
%
%  Created by Kyle Mandli on 2010-04-20.
%  Copyright (c) 2010 University of Washington. All rights reserved.
%
\documentclass[]{article}

% Use utf-8 encoding for foreign characters
\usepackage[utf8]{inputenc}

% Setup for fullpage use
\usepackage{fullpage}
\usepackage{verbatim}

% Uncomment some of the following if you use the features
%
% Running Headers and footers
%\usepackage{fancyhdr}

% Multipart figures
%\usepackage{subfigure}

% More symbols
\usepackage{amsmath}
\usepackage{amssymb}
\usepackage{latexsym}

% Surround parts of graphics with box
\usepackage{boxedminipage}

% Package for including code in the document
% \usepackage{listings}

% If you want to generate a toc for each chapter (use with book)
% \usepackage{minitoc}

% This is now the recommended way for checking for PDFLaTeX:
\usepackage{ifpdf}

\ifpdf
\usepackage[pdftex]{graphicx}
\else
\usepackage{graphicx}
\fi
% 
% % Commands 
% \include{commands}

% Graphics
\DeclareGraphicsRule{.tif}{png}{.png}{`convert #1 `dirname #1`/`basename #1 .tif`.png}
\usepackage{epstopdf}

\title{Riemann Solver Comparison:  Point-wise vs. Vector Evaluation}
\author{  }
\date{}

\begin{document}

\ifpdf
\DeclareGraphicsExtensions{.pdf, .jpg, .tif}
\else
\DeclareGraphicsExtensions{.eps, .jpg}
\fi

\maketitle

\section{Introduction}
The current version of Clawpack evaluates the Riemann solver by sending a vector of quantities into the function that will evaluate the Riemann solution and give back a vector of waves and wave speeds for each grid cell.  This solve requires a loop over all grid cells inside of the Riemann problem.  In order to simplify the code needed to solve these problems and provide a simpler interface to the Riemann solver, it has been suggested that a point-wise function call, where the Riemann solver only solves for one interface, be implemented.

The goal of this project is to understand the inefficiencies and consequences of moving to a point-wise Riemann solver from the current vector based one. 

\section{Algorithm Background}
A Riemann problem is defined as a hyperbolic PDE such as
\begin{equation}
    q_t + f(q)_x = 0,
\end{equation}
where $q$ is a vector of size $m$ and $f(q)$ is the flux function dependent on the state vector $q$, and piecewise constant initial data with a jump discontinuity.
\begin{equation}
    q(x,0) = \left \{ \begin{aligned}
    &q_\ell ~~~~~ \text{if } x < 0,\\
    &q_r ~~~~~ \text{if } x > 0. \end{aligned} \right .
\end{equation}
The solution to this problem includes the determination of a set of waves $\mathcal{W^p} \in \mathcal{R}^m$ along with their associated speeds $s^p$.  The Riemann solver then needs to accept the left and right values of $q$ to either side of the Riemann problem and output the set of waves $W^p$ and speeds $s^p$ then determine how we update to the next time.  This procedure is done for each grid cell interface giving a vector of values for $\mathcal{W}$ and $s$.  

\section{Goals}
There are a few known sources of overhead increase in moving from the vectorized version of the code to the point-wise version:
\begin{enumerate}
    \item Function call overhead - The number of function calls increases proportional to $mx+1$ which can be considerable since this is done many times for the life of the simulation.  Function call overhead has been evaluated for simple functions (such as a function that takes the square root of a number) but a compiler can identify these types of functions and optimize accordingly.  It is not clear if a compiler can always identify the Riemann solvers and optimize them well without a lot of ``hand holding''.
    \item Slicing of arrays - Array slicing is needed to pass the arguments in and assign them to the output correctly, i.e.
\begin{verbatim}
pw_rp(q(:,i),q(:,i),aux(:,i-1),aux(:,i),wave(:,:,i),s(:,i))
\end{verbatim}
    Windowing using pointers is another possible avenue but has not been looked into.  There are large penalties for slicing sometimes, depending on how one goes about it.
\end{enumerate}

\bibliographystyle{plain}
\bibliography{/Users/mandli/Documents/papers/database.bib}
\end{document}
